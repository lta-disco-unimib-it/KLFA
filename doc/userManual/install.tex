\chapter{Installing and Compiling KLFA}

\section{Installing a compiled version of KLFA}
\label{sec:installKLFA}

If you received the KLFA distribution zip (something like
klfa-201010141601.zip), just uncompress it in the location you prefer, 
e.g. \texttt{/home/fabrizio/Pro\-grams/klfa\-201010141601}.
  
Once you uncompressed it you just need to do the following commands:

	1) (if using Linux or OSX) make scripts executables
		e.g. 
		\begin{verbatim}
		chmod a+x /home/fabrizio/Programs/klfa-201010141601/bin/*
		\end{verbatim}

	2) (for any OS) set the environment variable KLFA$\_$HOME to point to the 
		folder where you installed klfa, e.g. 
		\texttt{/home/fabrizio/Programs/klfa-201010141601/}
		
		If you are using Linux or OSX with the BASH shell you could add the 
		following line to file \texttt{\$HOME.bashrc}:
		
		\begin{verbatim}
		export KLFA_HOME=/home/fabrizio/Programs/klfa-201010141601/
		\end{verbatim}

	Change the path according to your KLFA installation path.
	
	3) (for any OS) add the bin folder in KLFA$\_$HOME to the PATH environment 
		variable.
		
		If you are using Linux or OSX with the BASH shell you could add the 
		following line to file .bashrc (change the path according to your path):
		
		\begin{verbatim}
		export PATH=$PATH:/home/fabrizio/Programs/klfa-201010141601/bin/
		\end{verbatim}
		
		
		You can check if the previous command succeeded by running the following
		command and checking that you have an output similar to the one 
		reported below:
		
		\begin{verbatim}
		$ which klfaCsvAnalysis.sh
		/home/fabrizio/Programs/klfa-201010141601/bin//klfaCsvAnalysis.sh
		\end{verbatim}


		Check if klfa is correctly installed by running:
		\begin{verbatim}
		$ klfaCsvAnalysis.sh 
		\end{verbatim}
		
		The command will output KLFA command help. Like in the following paragraph:
		\begin{verbatim}
This program builds models of the application behavior by analyzing a trace
file. The trace file must be a collection of lines, each one in the format
COMPONENT,EVENT[,PARAMETER]. 
Multiple traces can be defined in a file, to
separate a trace from another put a line with the | symbol. 
Usage : 
	it.unimib.disco.lta.alfa.klfa.LogTraceAnalyzer [options] <analysisType> <phase>
		<valueTranformersConfigFile> <preprocessingRules> <traceFile>
	\end{verbatim}
	
	
KLFA includes several programs and utilities described in the following
Sections. The most common utilities can be run by using the shell scripts in
\texttt{KLFA$\_$HOME/bin}
	
We suggest to go through the examples in folder \texttt{KLFA$\_$HOME/examples}
to understand how to use KLFA. Some examples are described in
Chapter~\ref{ch:examples}, others are described in the file \texttt{README.txt}
that you find in each example folder.



\section{Compiling KLFA from a source distribution}
\label{sec:compileKLFA}

If you received a source distribution zip of KLFA (something like 
klfa-src-201010141601.zip), uncompress it in the location you prefer, 
e.g. /home/fabrizio/Programs/klfa-src\-201010141601.


In order to compile an installable version of klfa from sources run the following
command within the folder where you uncompressed klfa:
\begin{verbatim}
	ant distribution
\end{verbatim}	
	
so you could do:
\begin{verbatim}
	cd 	/home/fabrizio/Programs/klfa-src-201010141601
	ant distribution
\end{verbatim}	
	
The command will create the KLFA distribution zip in the dist folder.
e.g.
\texttt{/home/fabrizio/Programs\-/klfa-src\-201010141601/dist/klfa\-201010141601.zip}

After creating the distribution zip you can follow the commands described in
Section~\ref{sec:installKlfa}.

\section{Compiling KLFA from CVS}
\label{sec:compileklfaCVS}

In order to install the head version of klfa stored on the UniMiB CVS repository
you need to download the following CVS modules:
\begin{itemize}
  \item LogFileAnalysis-LFA
  \item BCT (you need to download the TPTPIntegration branch)
\end{itemize}

LogFileAnalysis-LFA is klfa. BCT
provides the libraries to infer automata.

The first step is the compilation of klfa dependencies. To do so run
\begin{verbatim}
ant buildDependencies
\end{verbatim}

The command will create the library \textit{bct.jar} in
folder \textit{lib}.

Next step is to run the command

\begin{verbatim}
	ant distribution
\end{verbatim}	

This command builds the klfa distribution zip. Follow the instructions described
in Section~\ref{sec:installKLFA} to install KLFA.

Other klfa ant compilation options are described by the build.xml help. To see
the other compilation options just run
\begin{verbatim}
	ant
\end{verbatim}



\section{Installing SLCT}
\label{sec:installSLCT}

In order to identify event types AVA uses SLCT~\cite{VaarandiIPOM2003}. In order to install SLCT
you need to change your current directory to \texttt{src-native/slct-0.5} and
compile slct.

If you use Linux or OsX you can run the following commands

\begin{verbatim}
cd $AVA_HOME/../src-native/slct-0.5
gcc -o slct -O2 -D_LARGEFILE_SOURCE -D_FILE_OFFSET_BITS=64 slct.c
sudo mkdir /opt/slct-0.5
sudo mv slct /opt/slct-0.5
\end{verbatim}